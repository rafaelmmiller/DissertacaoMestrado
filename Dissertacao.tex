\documentclass[a4paper,12pt]{article}
\usepackage[utf8]{inputenc}
\usepackage[brazilian]{babel}
\usepackage{color}
\usepackage{graphicx}
\usepackage{subfig}
\usepackage[hmargin=2cm,vmargin=3.5cm,bmargin=2cm]{geometry}
\usepackage{cite}
\usepackage{float}
\usepackage{enumerate}
\usepackage{amsmath}
\usepackage{verbatim}

\begin{document}

\title{}

\author{Aluno: Rafael M. Miller NUSP.: 7581818}

\maketitle

\section{SPHENE - ideal Smooth Particle Hydrodynamics simulation of hEavy ion NuclEar collisions}

Acho melhor falar sobre a implementação do método aplicado às colisões de íons pesados mesmo. O que você acha Frederique?

\subsection{Método SHP}

História do método (preciso ler os artigos originais):
\begin{itemize}
  \item Gingold e Monaghan (1977), Lucy (1977)

  R.A. Gingold and J.J. Monaghan, ``Smoothed particle hydrodynamics: theory and application to non-spherical stars,'' Mon. Not. R. Astron. Soc., Vol 181, pp. 375–89, 1977.

  L.B. Lucy, ``A numerical approach to the testing of the fission hypothesis,'' Astron. J., Vol 82, pp. 1013–1024, 1977.
  \item usos atuais (?)
\end{itemize}

Explicar o processo de discretização através da função {\it kernel}.

Obter a equação de movimento da partícula SPH através da ação.

\subsection{Método de Integração de Runge-Kutta}

Nada muito aprofundado.

\subsection{Arquitetura do código}

Descrever todo a estrutura de classes do código, porque foi pensado assim e os padrões de projeto utilizados (o que são padrões de projeto? https://pt.wikipedia.org/wiki/Padr%C3%A3o_de_projeto_de_software)

\subsection{O algoritmo}

Um passo a passo do algoritmo utilizado. O fluxograma (eu nunca vi esse fluxograma bem descrito).

\subsection{Hadronização}

Essa parte eu ainda preciso estudar e implementar. Devo colocar nesse capítulo?

\subsection{Resultados}

Alguns resultados.





%\begin{thebibliography}{99}

%\bibitem{rhic}
%Relativistic Heavy Ion Collider website: \verb#http://www.bnl.gov/rhic/#

%\bibitem{lhc}
%Large Hadron Collider website: \verb#http://home.web.cern.ch/topics/large-hadron-collider#

%\bibitem{weinberg}
%Steven Weinberg. Três Primeiros Minutos, Os. Gradiva, 1987

%\bibitem{spherio}
%Y. Hama, T. Kodama, O. Socolowski, Jr., Braz. J. Phys. \textbf{35}, 24 (2005).

%\bibitem{Jaki}
%J. Noronha-Hostler, G. S. Denicol, J. Noronha, R. P. G. Andrade, F. Grassi, Phys. Rev. C \textbf{88}, 044916 (2013) [arXiv:1305.1981 [nucl-th]].

%\bibitem{gubser}
%S. S. Gubser, Phys. Rev. D \textbf{82}, 085027 (2010) [arXiv:1006.0006 [hep-th]].

%\bibitem{bjorken}
%J. D. Bjorken, Phys. Rev. D \textbf{27}, 140 (1983).

%\bibitem{noronha}
%H. Marrochio, J. Noronha, G. S. Denicol, M. Luzum, S. Jeon, C. Gale [arXiv:1307.6130 [nucl-th]]

%\bibitem{nexus}
%T. Pierog, M. Bleicher, K. Mikhailov, K. Werner, I. Karpenko. Phys. Rev. C \textbf{82}, 044904(2010). [arXiv:1004.0805 [nucl-th]].

%\bibitem{glauber}
%H.-J. Drescher, Y. Nara, Phys. Rev. \textbf{C 75}, 034905 (2007); 76, 041903 (2007)

%\bibitem{1tubo1}
%Y. Hama, R. P. G. Andrade, F. Grassi, W. L. Qian, Nonlin. Phenom. Complex. Syst. \textbf{12}, 466(2009). [arXiv:0911.0811 [hep-ph]].

%\bibitem{1tubo2}
%R. P. G. Andrade, F. Grassi, Y. Hama, W. L. Qian, Physics Letters. B, \textbf{712} [arXiv:1008.4612 [nucl-th]].

%\bibitem{fernando}
%F. G. Gardim, F. Grassi, M. Luzum , J-Y. Ollitrault, Phys. Rev. C \textbf{85}, 024908 (2012) [arXiv:1111.6538 [nucl-th]]

%\bibitem{cooperfrye}
%F. Cooper and G. Frye, Phys. Rev. D \textbf{10}, 186 (1974).

%\bibitem{Romatschke}
%Romatschke, P., Int.J.Mod.Phys.\textbf{19} 1-53 (2010), arXiv:0902.3663v3 [hep-ph]

%\bibitem{nobel}
%``The Nobel Prize in Physics 2013". Nobelprize.org. Nobel Media AB 2014. Web. 17 Jul 2014. \verb#http://www.nobelprize.org/nobel_prizes/physics/laureates/2013/#

%\end{thebibliography}

\end{document}
